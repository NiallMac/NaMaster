\documentclass[a4paper,10pt]{article}
\usepackage{graphicx} % Include figure files
\usepackage{fullpage}
\usepackage[usenames,dvipsnames,svgnames,table]{xcolor}
\usepackage{amssymb,amsmath} 
\usepackage[colorlinks=true,linkcolor=Blue,linktoc=page,citecolor=Red,urlcolor=BrickRed]{hyperref}
%\usepackage{color}
%\usepackage[utf8x]{inputenc}
\usepackage{enumitem}
\usepackage{hyperref}
\usepackage{listings}
%\usepackage{makeidx}
\usepackage{fancyhdr}
\usepackage{titlesec}
\usepackage{epigraph}
\usepackage{titling}
%\renewcommand{\familydefault}{\sfdefault}
\newcommand{\apj}{ApJ}
\newcommand{\apjs}{ApJS}
\newcommand{\mnras}{MNRAS}

\pretitle{\begin{flushleft}\LARGE\bfseries\sffamily}
\posttitle{\par\end{flushleft}\vskip 0.5em\vspace{-7ex}}
\predate{\begin{flushleft}\sffamily\large}
\postdate{\par\end{flushleft}\vskip 0.5em\vspace{-2ex}}
\titleformat{\section}[hang]{\bfseries\sffamily\Large}
{\thesection}
{12pt}{\filright}

\titleformat{\subsection}[hang]{\bfseries\sffamily}
{\thesubsection}
{12pt}{\filright}

\titleformat{\subsubsection}[hang]{\bfseries}
{\thesubsubsection}
{12pt}{\filright}

\newcommand{\Tr}{{\rm Tr}}
\newcommand{\nv}{\hat{\bf n}}
\newcommand{\wtj}[6]{\left(\begin{array}{ccc} #1 & #2 & #3\\#4 & #5 & #6\end{array} \right)}


\title{NaMaster: Scientific Documentation}

\begin{document}
\maketitle
\noindent\makebox[\linewidth]{\rule{\textwidth}{1pt}}
\tableofcontents
\noindent\makebox[\linewidth]{\rule{\textwidth}{1pt}}

\section{Introduction}
NaMaster is a C library, python module and standalone program to compute the pseudo-$C_\ell$ estimator of the angular power spectrum between two masked and contaminated fields (this is also the so-called ``MASTER'' algorithm). The contents of this scientific documentation describe the algorithm, drawing heavily from the methods presented by \cite{2002ApJ...567....2H} and \cite{2017MNRAS.465.1847E}, extending their results to arbitrary cross-correlations between spin-0 and spin-2 fields (see also \cite{2003ApJS..148..161K}).

\section{Generalities and SHTs}
Let ${\bf a}(\nv)$ be a spin-$s_a$ quantity defined on the sphere. Then we define its spherical harmonic coefficients as:
\begin{equation}
 {\bf a}_{\ell m}\equiv{\rm SHT}({\bf a}(\nv))^{s_a}_{\ell m}\equiv\int d\nv\,\hat{\sf Y}^{s_a\dag}_{\ell m}(\nv)\,{\bf a}(\nv),\hspace{12pt}
 {\bf a}(\nv)={\rm SHT}^{-1}({\bf a}_{\ell m})^{s_a}_{\nv}\equiv\sum_{\ell m}\,\hat{\sf Y}^{s_a}_{\ell m}(\nv)\,{\bf a}_{\ell m}.
\end{equation}
Note that here we will use a vector notation, such that for a complex spin-$s_a$ field $a$ we form the vector ${\bf a}\equiv({\rm Re}(a),{\rm Im}(a))$. The harmonic coefficients above are decomposed in a similar manner into $E$ and $B$ modes: ${\bf a}_{\ell m}\equiv(a^E_{\ell m},a^B_{\ell m}$. The spherical harmonic operators $\hat{\sf Y}^s$ are therefore matrix that we define in the following subsection.

\subsection{Spin-weighed spherical harmonics}
  Let $\eth$ and $\bar{\eth}$ be the following complex differential operators defined on the sphere when acting on a spin-$s$ quantity $f_s$:
  \begin{equation}
    \eth f_s\equiv-(\sin\theta)^s\left(\partial_\theta+i\frac{\partial_\phi}{\sin\theta}\right)(\sin\theta)^{-s}\,f_s(\theta,\phi),\hspace{12pt}
    \bar{\eth} f_s\equiv-(\sin\theta)^{-s}\left(\partial_\theta-i\frac{\partial_\phi}{\sin\theta}\right)(\sin\theta)^{s}\,f_s(\theta,\phi).
  \end{equation}
  The following properties can be easily derived for the action of these operators:
  \begin{itemize}
    \item If $f_s$ is a spin-$s$ quantity, $(f_s)^*$ is a spin-$(-s)$ quantity.
    \item $\eth f_s$ is a spin-$(s+1)$ quantity, and $\bar{\eth} f_s$ is a spin-$(s-1)$ quantity.
    \item $(\eth^nf_s)^*=\bar{\eth}^n(f_s)^*$
    \item $\eth(f\,g)=f\eth g+g\eth f$
    \item $\eth^2(f\,g)=f\eth^2g+g\eth^2f+\eth f\eth g$
  \end{itemize}
  
  We start by defining the spin-weighed spherical harmonics with spin $s\geq0$:
  \begin{equation}
    _sY_{\ell m}\equiv \alpha_{\ell,s} \eth^s Y_{\ell m},\hspace{6pt}
    _{-s}Y_{\ell m}\equiv \alpha_{\ell,s} (-1)^s\bar{\eth}^s Y_{\ell m},\hspace{6pt}
    \alpha_{l,s}\equiv\sqrt{\frac{(\ell-s)!}{(\ell+s)!}},
  \end{equation}
  which have the property: $(_sY_{\ell m})^*=(-1)^{s+m}\,_{-s}Y_{\ell-m}$. We then define the $E$-mode and $B$-mode spherical harmonic vectors as:
  \begin{align}
    &_s{\bf Y}^E_{\ell m}\equiv {\bf D}^E_sY_{\ell m}
     \equiv\frac{\alpha_{\ell,s}}{2}\left(\begin{array}{c}
                              \eth^s+\bar{\eth}^s\\
                              -i(\eth^s-\bar{\eth}^s)
                            \end{array}\right)Y_{\ell m}
     \equiv\frac{1}{2}\left(\begin{array}{c}
                              _sY_{\ell m}+(-1)^s\,_{-s}Y_{\ell m}\\
                              -i(_sY_{\ell m}-(-1)^s\,_{-s}Y_{\ell m})
                            \end{array}\right) \\
    &_s{\bf Y}^B_{\ell m}\equiv {\bf D}^B_sY_{\ell m}
     \equiv\frac{\alpha_{\ell,s}}{2}\left(\begin{array}{c}
                              i(\eth^s-\bar{\eth}^s)\\
                              \eth^s+\bar{\eth}^s
                            \end{array}\right)Y_{\ell m}
     \equiv\frac{1}{2}\left(\begin{array}{c}
                              i(_sY_{\ell m}-(-1)^s\,_{-s}Y_{\ell m})\\
                              _sY_{\ell m}+(-1)^s\,_{-s}Y_{\ell m}
                            \end{array}\right),
  \end{align}
  which also defines the differential operators ${\bf D}^{E,B}_s$. These functions, for  $s=0$ are simply ${\bf D}^{E}_0=(Y_{\ell m},0)$ and ${\bf D}^{B}_0=(0,Y_{\ell m})$.
  
  The matrix operator $\hat{\sf Y}^s_{\ell m}$ is then defined as having $_s{\bf Y}^{E,B}_{\ell m}$ as columns:
  \begin{equation}
    \hat{\sf Y}^s_{\ell m}\equiv\left(_s{\bf Y}^E_{\ell m},_s{\bf Y}^B_{\ell m},\right)
    \equiv\frac{1}{2}
    \left(\begin{array}{cc}
            _sY_{\ell m}+(-1)^s\,_{-s}Y_{\ell m}     & i(_sY_{\ell m}-(-1)^s\,_{-s}Y_{\ell m})\\                
            -i(_sY_{\ell m}-(-1)^s\,_{-s}Y_{\ell m}) & _sY_{\ell m}+(-1)^s\,_{-s}Y_{\ell m}                
          \end{array}\right)
  \end{equation}

  The matrices $\hat{\sf Y}^s_{\ell m}$ satisfy the following relations:
  \begin{align}
    \hat{\sf Y}^{s\dag}_{\ell m}&=(-1)^{m+s}\hat{\sf Y}^{-s}_{\ell -m}\\
    \int d\nv \hat{\sf Y}^s_{\ell m}\hat{\sf Y}^{s\dag}_{\ell'm'}&=\hat{\sf 1}\delta_{\ell\ell'}\delta_{mm'}\\
    \hat{\sf D}^s_{{\bf l}{\bf l}_1{\bf l}_2}&\equiv\int d\nv \left(\hat{\sf Y}^{s\dag}_{\bf l}(\nv)\,\hat{\sf Y}^{s}_{{\bf l}_1}(\nv)\right)\,\hat{\sf Y}^0_{{\bf l}_2}(\nv),\\
    &=(-1)^{s+m}\sqrt{\frac{(2\ell+1)(2\ell_1+1)(2\ell_2+1)}{4\pi}}\wtj{\ell}{\ell_1}{\ell_2}{-m}{m_1}{m_2}\wtj{\ell}{\ell_1}{\ell_2}{s}{-s}{0}\,\hat{\sf d}_{\ell+\ell_1+\ell_2},\\
    \hspace{12pt}&\hat{\sf d}^2_n=\frac{1}{2}\left(\begin{array}{cc}
                                       1+(-1)^n & -i[1-(-1)^n]\\
                                       i[1-(-1)^n] & 1+(-1)^n
                                     \end{array}\right),
  \end{align}
  where we have abbreviated the pair $(\ell,m)$ as ${\bf l}$.

  Finally, the following orthogonality relation for the Wigner 3$j$ symbols is useful:
  \begin{equation}
    \sum_{mm_1}\wtj{\ell}{\ell_1}{\ell_2}{m}{m_1}{m_2}\wtj{\ell}{\ell_1}{\ell_3}{m}{m_1}{m_3}=\frac{\delta_{\ell_2\ell_3}\delta_{m_2m_3}}{2\ell_2+1}
  \end{equation}

\subsection{$E$ and $B$ mode purification}
  We define a field ${\bf f}$ to be a $B$ ($E$) mode if $({\bf D}^{E(B)}_s)^\dag{\bf f}=0$. At the same time, and under the definition of the dot product:
  \begin{equation}
    ({\bf f},{\bf g})\equiv\int d\nv\,{\bf f}^\dag{\bf g},
  \end{equation}
  we define a {\sl pure} $B$ ($E$) mode as a field that is orthogonal to all $E$ ($B$) modes.
  
  Since ${\bf D}^{E\dag}_s{\bf D}^B_s=0$, one can always generate a $B$ ($E$) mode by applying ${\bf D}^{B(E)}_s$ to a scalar field. It is then possible to show that $E$ and $B$ modes thus defined are orthogonal in the full sky:
  \begin{equation}
    ({\bf D}^E_s\phi,{\bf D}^B_s\psi)=\int d\nv\,({\bf D}^E_s\phi)^\dag{\bf D}^B_s\psi=0
  \end{equation}
  This can be done by integrating by parts and noting that the celestial sphere has no boundaries. On a cut sky, however, and for $s=2$, this is only true if the fields satisfy Neumann and Dirichlet boundary conditions simultaneously (i.e. vanishing value and first derivative on the boundary of the cut sky region).
  
  Let $w(\nv)$ be a sky window function defining the sky region to be analyzed (and the weight to be applied in each pixel). The standard pseudo $B$-mode of a field ${\bf P}$ is then given by
  \begin{equation}
    \tilde{B}_{\ell m}=\int d\nv\,w(\nv)\left(_s{\bf Y}^{B}_{\ell m}(\nv)\right)^\dag{\bf P}=\int d\nv\,w(\nv)({\bf D}^B_sY_{\ell m})^\dag{\bf P}(\nv),
  \end{equation}
  Now, since ${\bf D}^B_sY_{\ell m}$ is a $B$-mode, in the absence of $w$ this expresion would correspond to a projection that filters out all the $E$-modes from ${\bf P}$. However, $w(\nv){\bf D}^B_sY_{\ell m}$ is not a $B$-mode, and therefore $\tilde{B}_{\ell m}$ receives contributions from ambiguous $E$ modes (which then propagate into the variance of the pseudo-$C_\ell$ estimator of the power spectrum).
  
  The idea behind $B$-mode purification is to move $w$ to the right of ${\bf D}^B_s$, defining the field:
  \begin{equation}
   B^p_{\ell m}=\int d\nv\left({\bf D}^B_2(w\,Y_{\ell m})\right)^\dag\,{\bf P}(\nv).
  \end{equation}
  Since ${\bf D}^B_2(wY_{\ell m})$ is a $B$-mode quantity, $B^p_{\ell m}$ should receive contributions only from $B$-modes.
  
  Expanding ${\bf D}^B_2(wY_{\ell m})$, we can write $B^p_{\ell m}$ as:
  \begin{equation}
    B^p_{\ell m}=
    \left(\tilde{P}_2\right)^B_{\ell m}+
    2\frac{\alpha_{\ell,2}}{\alpha_{\ell,1}}\left(\tilde{P}_1\right)^B_{\ell m}+\alpha_{\ell,2}\left(\tilde{P}_2\right)^B_{\ell m},
  \end{equation}
  where $(f)^B_{\ell m}$ stands for the $B$-mode of field $f$, and we have defined the fields $\tilde{P}_n=(\eth^nw)^*(Q+iU)$, where $Q$ and $U$ are the real and imaginary parts of the field $P$.
  
  Note that the derivatives of $w$ can be computed as:
  \begin{equation}
    w
    \rightarrow w_{\ell m}={\rm SHT}(w)
    \rightarrow\,\left\{_nw^E_{\ell m}=(-1)^Hw_{\ell m}/\alpha_{\ell,n},\,_nw^B_{\ell m}=0\right\}
    \rightarrow \eth^nw={\rm SHT}^{-1}(\{\,_nw^E_{\ell m},\,_nw^B_{\ell m}\})
  \end{equation}


\section{Contaminant cleaning}
  Let ${\bf a}$ be a random field defined on the sphere, let $v(\nv)$ a mask for ${\bf a}$ and let ${\bf f}^i$ be a set of $N_a$ contaminants of ${\bf a}$ such that the observed version of ${\bf a}$ be:
  \begin{equation}
    {\bf d}_a(\nv)={\bf a}^v(\nv)+\sum_{i=1}^{N_a}\alpha_i{\bf f}^i(\nv),
  \end{equation}
  where ${\bf a}^v(\nv)=v(\nv)\,{\bf a}(\nv)$ (note that we have implicitly applied the same mask to ${\bf f}^i$). The best-fit value for the coefficients $\alpha_i$ assuming the same weights for all points in ${\bf d}_a$ can be found as
  \begin{equation}
    \tilde{\alpha}_i=\sum_jM_{ij}\int d\nv\,{\bf f}^{j\dag}(\nv)\,{\bf d}_a(\nv),\hspace{12pt}
    (\hat{\sf M}^{-1})_{ij}\equiv\int d\nv\,{\bf f}^{i\dag}(\nv)\,{\bf f}^j(\nv).
  \end{equation}
  Thus we can find a cleaned version of ${\bf a}$ as:
  \begin{align}
    \tilde{\bf a}(\nv)&\equiv{\bf d}_a(\nv)-{\bf f}^i(\nv)\,M_{ij}\int d\nv'{\bf f}^{j\dag}(\nv'){\bf d}_a(\nv')\\
                      &={\bf a}^v(\nv)-{\bf f}^i(\nv)\,M_{ij}\int d\nv'{\bf f}^{j\dag}(\nv'){\bf a}^v(\nv'),
  \end{align}
  where there is an implicit summation sign over $i$ and $j$ (we will omit these from now on).
  
  The harmonic coefficients of the cleaned and masked field are:
  \begin{equation}
   \tilde{\bf a}_{\ell m}={\bf a}^v_{\ell m}-{\bf f}^i_{\ell m}M_{ij}\sum_{\ell' m'}{\bf f}^{j\dag}_{\ell' m'}{\bf a}^{v}_{\ell' m'}.
  \end{equation}
  From now on we will simplify the notation by abbreviating the pair $\ell m$ as ${\bf l}$, so that the previous equation reads:
  \begin{equation}
   \tilde{\bf a}_{\bf l}={\bf a}^v_{\bf l}-{\bf f}^i_{\bf l}M_{ij}\sum_{{\bf l}'}{\bf f}^{j\dag}_{{\bf l}'}{\bf a}^{v}_{{\bf l}'}.
  \end{equation}

  The harmonic coefficients for the masked field can be related to those of the unmasked one and the mask $v$ (understood as a spin-0 field) as:
  \begin{equation}
    {\bf a}^v_{\bf l}=\sum_{{\bf l}_1{\bf l}_2}\hat{\sf D}^{s_a}_{{\bf l}{\bf l}_1{\bf l}_2}{\bf a}_{{\bf l}_1}v_{{\bf l}_2}.
  \end{equation}

\section{Pseudo-$C_\ell$ estimators with mode deprojection}  
  In what follows, for two fields ${\bf a}$ and ${\bf b}$ we will define their observed power spectrum as:
  \begin{equation}
    \tilde{\sf C}^{ab}_{\ell}\equiv\frac{1}{2\ell+1}\sum_m{\bf a}_{\ell m}{\bf b}^\dag_{\ell m}.
  \end{equation}
  This must not be confused with the true power spectrum defined as an ensemble average for isotropic fields:
  \begin{equation}
    \langle{\bf a}_{{\bf l}}{\bf b}^\dag_{{\bf l}'}\rangle\equiv\hat{\sf C}^{ab}_\ell\delta_{\ell\ell'}\delta_{mm'}.
  \end{equation}


  Now, let $\tilde{\bf a}$ and $\tilde{\bf b}$ be the contaminant-cleaned versions of two random fields ${\bf a}$ and ${\bf b}$ with contaminants ${\bf f}^i$ and ${\bf g}^j$ and masks $v$ and $w$ respectively, and let us define
  \begin{equation}
    (\hat{\sf N}^{-1})_{ij}\equiv\int d\nv\, {\bf g}^{i\dag}(\nv)\,{\bf g}^j(\nv).
  \end{equation}

  The observed power spectrum of the contaminant-cleaned maps can be written as:
  \begin{align}\nonumber
   \tilde{\sf C}^{\tilde{a}\tilde{b}}_\ell=&
   \frac{1}{2\ell+1}\sum_m{\bf a}^v_{\ell m}{\bf b}^{w\dag}_{\ell m}-\frac{N^*_{ij}}{2\ell+1}\sum_m\sum_{{\bf l}'}{\bf a}^v_{{\bf l}}{\bf b}^{w\dag}_{{\bf l}'}{\bf g}^{j}_{{\bf l}'}{\bf g}^{i\dag}_{\bf l}-\\
   &-\frac{M_{ij}}{2\ell+1}\sum_m\sum_{{\bf l}'}{\bf f}^{i}_{{\bf l}}{\bf f}^{j\dag}_{{\bf l}'}{\bf a}^v_{{\bf l}'}{\bf b}^{w\dag}_{\bf l}+\frac{M_{ij}N^*_{pq}}{2\ell+1}\sum_m\sum_{{\bf l}'{\bf l}''}{\bf f}^{i}_{{\bf l}}{\bf f}^{j\dag}_{{\bf l}'}{\bf a}^v_{{\bf l}'}{\bf b}^{w\dag}_{{\bf l}''}{\bf g}^{q}_{{\bf l}''}{\bf g}^{p\dag}_{\bf l}.
  \end{align}
  
  In order to compute the bias of $\tilde{\sf C}^{\tilde{a}\tilde{b}}_\ell$ with respect to $\hat{\sf C}^{ab}_\ell$, we need to compute the ensemble average of the former, which we will write as:
  \begin{equation}
    \left\langle\tilde{\sf C}^{\tilde{a}\tilde{b}}_\ell\right\rangle=\hat{\sf F}^1_\ell-\hat{\sf F}^2_\ell-\hat{\sf F}^3_\ell+\hat{\sf F}^4_\ell,
  \end{equation}
  where:
  \begin{align}
    \hat{\sf F}^1_\ell&\equiv\frac{1}{2\ell+1}\sum_m\langle{\bf a}^v_{\ell m}{\bf b}^{w\dag}_{\ell m}\rangle,\hspace{12pt}
    \hat{\sf F}^2_\ell\equiv\frac{N^*_{ij}}{2\ell+1}\sum_m\sum_{{\bf l}'}\langle{\bf a}^v_{{\bf l}}{\bf b}^{w\dag}_{{\bf l}'}{\bf g}^{j}_{{\bf l}'}{\bf g}^{i\dag}_{\bf l}\rangle\\
    \hat{\sf F}^3_\ell&\equiv\frac{M_{ij}}{2\ell+1}\sum_m\sum_{{\bf l}'}\langle{\bf f}^{i}_{{\bf l}}{\bf f}^{j\dag}_{{\bf l}'}{\bf a}^v_{{\bf l}'}{\bf b}^{w\dag}_{\bf l}\rangle,\hspace{12pt}
    \hat{\sf F}^4_\ell\equiv\frac{M_{ij}N^*_{pq}}{2\ell+1}\sum_m\sum_{{\bf l}'{\bf l}''}\langle{\bf f}^{i}_{{\bf l}}{\bf f}^{j\dag}_{{\bf l}'}{\bf a}^v_{{\bf l}'}{\bf b}^{w\dag}_{{\bf l}''}{\bf g}^{q}_{{\bf l}''}{\bf g}^{p\dag}_{\bf l}\rangle.
  \end{align}
  We will now compute the ensemble average of each of these terms.
  
  \subsection{$\hat{\sf F}^1_\ell$}
    \begin{align}\nonumber
      \hat{\sf F}^1_\ell&=\frac{1}{2\ell+1}\sum_{m{\bf l}_{1,2,3,4}}v_{{\bf l}_2}w^*_{{\bf l}_4}\hat{\sf D}^{s_a}_{{\bf l}{\bf l}_1{\bf l}_2}\langle{\bf a}_{{\bf l}_1}{\bf b}^\dag_{{\bf l}_3}\rangle\hat{\sf D}^{s_b\dag}_{{\bf l}{\bf l}_3{\bf l}_4}\\\nonumber
      &=\frac{1}{2\ell+1}\sum_{m{\bf l}_{1,2,3}}v_{{\bf l}_2}w^*_{{\bf l}_3}\hat{\sf D}^{s_a}_{{\bf l}{\bf l}_1{\bf l}_2}\hat{\sf C}^{ab}_{{\bf l}_1}\hat{\sf D}^{s_b\dag}_{{\bf l}{\bf l}_1{\bf l}_3}\\\nonumber
      &=\frac{1}{2\ell+1}\sum_{\ell_1{\bf l}_{2,3}}v_{{\bf l}_2}w^*_{{\bf l}_3}\frac{(2\ell+1)(2\ell_1+1)}{4\pi}\sqrt{(2\ell_2+1)(2\ell_3+1)}\wtj{\ell}{\ell_1}{\ell_2}{s_a}{-s_a}{0}\wtj{\ell}{\ell_1}{\ell_3}{s_b}{-s_b}{0}\\\nonumber
      &\hspace{50pt}\hat{\sf d}^{s_a}_{\ell+\ell_1+\ell_2}\hat{\sf C}^{ab}_{\ell_1}\hat{\sf d}^{s_b\dag}_{\ell+\ell_1+\ell_3}\sum_{mm_1}\wtj{\ell}{\ell_1}{\ell_2}{-m}{m_1}{m_2}\wtj{\ell}{\ell_1}{\ell_3}{-m}{m_1}{m_3}\\\label{eq:master_transformation}
      &=\sum_{\ell_1\ell_2}\frac{(2\ell_1+1)(2\ell_2+1)}{4\pi}\tilde{C}^{vw}_{\ell_2}\wtj{\ell}{\ell_1}{\ell_2}{s_a}{-s_a}{0}\wtj{\ell}{\ell_1}{\ell_2}{s_b}{-s_b}{0}\hat{\sf d}^{s_a}_{\ell+\ell_1+\ell_2}\hat{\sf C}^{ab}_{\ell_1}\hat{\sf d}^{s_b\dag}_{\ell+\ell_1+\ell_2}
    \end{align}
    
    For $v=w=1$ this reduces to $\tilde{C}^{vw}_\ell=4\pi\delta_{\ell0}$ and:
    \begin{align}\nonumber
      \hat{\sf F}^1_\ell
      &=\sum_{\ell_1\ell_2}(2\ell_1+1)(2\ell_2+1)\delta_{\ell_20}\wtj{\ell}{\ell_1}{\ell_2}{s_a}{-s_a}{0}\wtj{\ell}{\ell_1}{\ell_2}{s_b}{-s_b}{0}\hat{\sf d}^{s_a}_{\ell+\ell_1+\ell_2}\hat{\sf C}^{ab}_{\ell_1}\hat{\sf d}^{s_b\dag}_{\ell+\ell_1+\ell_2}\\\nonumber
      &=\sum_{\ell_1}(2\ell_1+1)\wtj{\ell}{\ell_1}{0}{s_a}{-s_a}{0}\wtj{\ell}{\ell_1}{0}{s_b}{-s_b}{0}\hat{\sf d}^{s_a}_{\ell+\ell_1}\hat{\sf C}^{ab}_{\ell_1}\hat{\sf d}^{s_b\dag}_{\ell+\ell_1}\\\nonumber
      &=\sum_{\ell_1}(2\ell_1+1)\delta_{\ell\ell_1}\frac{(-1)^{\ell-s_a}}{\sqrt{2\ell+1}}\delta_{\ell\ell_1}\frac{(-1)^{\ell-s_b}}{\sqrt{2\ell+1}}\hat{\sf d}^{s_a}_{\ell+\ell_1}\hat{\sf C}^{ab}_{\ell_1}\hat{\sf d}^{s_b\dag}_{\ell+\ell_1}\\\nonumber
      &=\hat{\sf d}^{s_a}_{2\ell}\hat{\sf C}^{ab}_{\ell}\hat{\sf d}^{s_b\dag}_{2\ell}\\\nonumber
      &=\hat{\sf C}^{ab}_{\ell}
    \end{align}


  \subsection{$\hat{\sf F}^2_\ell$}
    \begin{align}\nonumber
     \hat{\sf F}^2_\ell
     &=N^*_{ij}\int d\nv\,d\nv'\,\frac{\sum_{m{\bf l}'{\bf l}_{1,2,3,4}}}{2\ell+1}\hat{\sf Y}^{s_a\dag}_{\bf l}(\nv)\hat{\sf Y}^{s_a}_{{\bf l}_1}(\nv)\langle{\bf a}_{{\bf l}_1}{\bf b}^\dag_{{\bf l}_3}\rangle\hat{\sf Y}^{s_b\dag}_{{\bf l}_3}(\nv')\hat{\sf Y}^{s_b}_{{\bf l}'}(\nv'){\bf g}^j_{{\bf l}'}{\bf g}^{i\dag}_{\bf l}v_{{\bf l}_2}w^*_{{\bf l}_4}Y_{{\bf l}_2}(\nv)Y^*_{{\bf l}_4}(\nv')\\\nonumber
     &=N^*_{ij}\,\frac{\sum_m}{2\ell+1}\left\{\int d\nv\,v(\nv)\,\hat{\sf Y}^{s_a\dag}_{\bf l}(\nv)\left[\sum_{\ell_1m_1}\hat{\sf Y}^{s_a}_{{\bf l}_1}(\nv)\hat{\sf C}^{ab}_{\ell_1}\left(\int d\nv'\hat{\sf Y}^{s_b\dag}_{{\bf l}_1}(\nv'){\bf g}^j(\nv')w(\nv')\right)\right]{\bf g}^{i\dag}_{\bf l}\right\}\\\label{eq:bias_2}
     &=N^*_{ij}\,\frac{\sum_m}{2\ell+1}{\rm SHT}\left\{v(\nv){\rm SHT}^{-1}\left[\hat{\sf C}^{ab}_{\ell_1}{\rm SHT}\left(w\,{\bf g}^j\right)^{s_b}_{{\bf l}_1}\right]^{s_a}_{\nv}\right\}^{s_a}_{\bf l}{\bf g}^{i\dag}_{\bf l}
    \end{align}
    
    For $v=w=1$ this reduces to:
    \begin{align}\nonumber
     \hat{\sf F}^2_\ell
     &=N^*_{ij}\,\frac{\sum_m}{2\ell+1}{\rm SHT}\left\{{\rm SHT}^{-1}\left[\hat{\sf C}^{ab}_{\ell_1}{\rm SHT}\left({\bf g}^j\right)^{s_b}_{{\bf l}_1}\right]^{s_a}_{\nv}\right\}^{s_a}_{\bf l}{\bf g}^{i\dag}_{\bf l}\\\nonumber
     &=N^*_{ij}\,\hat{\sf C}^{ab}_{\ell}\frac{\sum_m{\bf g}^j_{\ell m}{\bf g}^{i\dag}_{\ell m}}{2\ell+1}\\\nonumber
     &=N^*_{ij}\,\hat{\sf C}^{ab}_{\ell}\tilde{\sf C}^{g^jg^i}_\ell
    \end{align}
    
  \subsection{$\hat{\sf F}^3_\ell$}
    \begin{align}\nonumber
     \hat{\sf F}^3_\ell
     &=M_{ij}\int d\nv\,d\nv'\,\frac{\sum_{m{\bf l}'{\bf l}_{1,2,3,4}}}{2\ell+1}{\bf f}^i_{\bf l}{\bf f}^{j\dag}_{{\bf l}'}\hat{\sf Y}^{s_a\dag}_{\bf l'}(\nv')\hat{\sf Y}^{s_a}_{{\bf l}_3}(\nv')\langle{\bf a}_{{\bf l}_3}{\bf b}^\dag_{{\bf l}_1}\rangle\hat{\sf Y}^{s_b\dag}_{{\bf l}_1}(\nv)\hat{\sf Y}^{s_b}_{{\bf l}}(\nv)v_{{\bf l}_4}w^*_{{\bf l}_2}Y_{{\bf l}_2}(\nv)Y^*_{{\bf l}_4}(\nv')\\\nonumber
     &=M_{ij}\frac{\sum_{m}}{2\ell+1}{\bf f}^i_{\bf l}\left\{\int d\nv\,w(\nv)\,\left[\sum_{\ell_1m_1}\left(\int d\nv'\,v(\nv'){\bf f}^{j\dag}(\nv')\hat{\sf Y}^{s_a}_{{\bf l}_1}(\nv')\right)\hat{\sf C}^{ab}_{\ell_1}\hat{\sf Y}^{s_b\dag}_{{\bf l}_1}(\nv)\right]\hat{\sf Y}^{s_b}_{{\bf l}}(\nv)\right\}\\\label{eq:bias_3}
     &=M_{ij}\frac{\sum_{m}}{2\ell+1}{\bf f}^i_{\bf l}\,{\rm SHT}\left\{w(\nv)\,{\rm SHT}^{-1}\left[\hat{\sf C}^{ab\dag}_{\ell_1}{\rm SHT}\left(v\,{\bf f}^j\right)^{s_a}_{{\bf l}_1}\right]^{s_b}_{\nv}\right\}^{s_b\dag}_{\bf l}
    \end{align}

    For $v=w=1$ this reduces to:
    \begin{align}\nonumber
     \hat{\sf F}^3_\ell
     &=M_{ij}\frac{\sum_{m}}{2\ell+1}{\bf f}^i_{\bf l}\,{\rm SHT}\left\{{\rm SHT}^{-1}\left[\hat{\sf C}^{ab\dag}_{\ell_1}{\rm SHT}\left({\bf f}^j\right)^{s_a}_{{\bf l}_1}\right]^{s_b}_{\nv}\right\}^{s_b\dag}_{\bf l}\\\nonumber
     &=M_{ij}\frac{\sum_{m}{\bf f}^i_{\ell m}{\bf f}^{j\dag}_{\ell m}}{2\ell+1}\hat{\sf C}^{ab}_{\ell}\\\nonumber
     &=M_{ij}\tilde{\sf C}^{f^if^j}_\ell\hat{\sf C}^{ab}_{\ell}
    \end{align}
    
  \subsection{$\hat{\sf F}^4_\ell$}
    \begin{align}\nonumber
     \hat{\sf F}^4_\ell
     &=\frac{M_{ij}N^*_{pq}}{2\ell+1}\int d\nv\,d\nv'\,\sum_{m{\bf l}'{\bf l}''{\bf l}_{1,2,3,4}}
     {\bf f}^i_{\bf l}{\bf f}^{j\dag}_{{\bf l}'}\hat{\sf Y}^{s_a\dag}_{\bf l'}(\nv)\hat{\sf Y}^{s_a}_{{\bf l}_1}(\nv)\langle{\bf a}_{{\bf l}_1}{\bf b}^\dag_{{\bf l}_3}\rangle\hat{\sf Y}^{s_b\dag}_{{\bf l}_3}(\nv')\hat{\sf Y}^{s_b}_{{\bf l}''}(\nv'){\bf g}^q_{{\bf l}''}{\bf g}^{p\dag}_{\bf l}v_{{\bf l}_2}w^*_{{\bf l}_4}Y_{{\bf l}_2}(\nv)Y^*_{{\bf l}_4}(\nv')\\\nonumber
     &=\frac{M_{ij}N^*_{pq}}{2\ell+1}\sum_{m}
     {\bf f}^i_{\bf l}\left\{\int d\nv\,v(\nv)\,{\bf f}^{j\dag}(\nv)\left[\sum_{\ell_1m_1}\hat{\sf Y}^{s_a}_{{\bf l}_1}(\nv)\hat{\sf C}^{ab}_{\ell_1}\left(\int d\nv'\,\hat{\sf Y}^{s_b\dag}_{{\bf l}_1}(\nv'){\bf g}^q(\nv')w(\nv')\right)\right]\right\}{\bf g}^{p\dag}_{\bf l}\\\label{eq:bias_4}
     &=M_{ij}N^*_{pq}
     \left\{\int d\nv\,v(\nv)\,{\bf f}^{j\dag}(\nv)\,{\rm SHT}^{-1}\left[\hat{\sf C}^{ab}_{\ell_1}{\rm SHT}\left(w\,{\bf g}^q\right)^{s_b}_{{\bf l}_1}\right]^{s_a}_{\nv}\right\}\tilde{\sf C}^{f^ig^p}_\ell
    \end{align}

    For $v=w=1$ this reduces to:
    \begin{align}\nonumber
     \hat{\sf F}^4_\ell
     &=M_{ij}N^*_{pq}\left\{\int d\nv\,{\bf f}^{j\dag}(\nv)\,{\rm SHT}^{-1}\left[\hat{\sf C}^{ab}_{\ell_1}{\rm SHT}\left({\bf g}^q\right)^{s_b}_{{\bf l}_1}\right]^{s_a}_{\nv}\right\}\tilde{\sf C}^{f^ig^p}_\ell\\\nonumber
     &=M_{ij}N^*_{pq}\left\{\int d\nv\,\sum_{{\bf l}_2}{\bf f}^{j\dag}_{{\bf l}_2}\hat{\sf Y}^{s_a\dag}_{{\bf l}_2}(\nv)\,\sum_{{\bf l}_1}\hat{\sf Y}^{s_a}_{{\bf l}_1}(\nv)\hat{\sf C}^{ab}_{\ell_1}{\bf g}^q_{{\bf l}_1}\right\}\tilde{\sf C}^{f^ig^p}_\ell\\\nonumber
     &=M_{ij}N^*_{pq}{\bf f}^{j\dag}_{{\bf l}_1}\hat{\sf C}^{ab}_{\ell_1}{\bf g}^q_{{\bf l}_1}\tilde{\sf C}^{f^ig^p}_\ell\\\nonumber
     &=M_{ij}N^*_{pq}\left[\sum_{\ell_1}(2\ell_1+1){\rm Tr}\left(\hat{\sf C}^{ab}_{\ell_1}\tilde{\sf C}^{g^qf^j}_{\ell_1}\right)\right]\tilde{\sf C}^{f^ig^p}_\ell
    \end{align}

  \subsection{Final form of the estimator}
    Putting together the results from Equations \ref{eq:master_transformation}, \ref{eq:bias_2}, \ref{eq:bias_3} and \ref{eq:bias_4}, we can write down an unbiased estimator for the pseudo-$C_\ell$ of the cut-sky maps free from contamination from ${\bf f}$ and ${\bf g}$:
    \begin{align}\nonumber
      \tilde{\sf C}^{ab}_\ell=&\tilde{\sf C}^{\tilde{a}\tilde{b}}_\ell+N^*_{ij}\,\frac{\sum_m}{2\ell+1}{\rm SHT}\left\{v(\nv){\rm SHT}^{-1}\left[\hat{\sf C}^{ab}_{\ell_1}{\rm SHT}\left(w\,{\bf g}^j\right)^{s_b}_{{\bf l}_1}\right]^{s_a}_{\nv}\right\}^{s_a}_{\bf l}{\bf g}^{i\dag}_{\bf l}+\\\nonumber
      &+M_{ij}\frac{\sum_{m}}{2\ell+1}{\bf f}^i_{\bf l}\,{\rm SHT}\left\{w(\nv)\,{\rm SHT}^{-1}\left[\hat{\sf C}^{ab\dag}_{\ell_1}{\rm SHT}\left(v\,{\bf f}^j\right)^{s_a}_{{\bf l}_1}\right]^{s_b}_{\nv}\right\}^{s_b\dag}_{\bf l}-\\
      &-M_{ij}N^*_{pq}\left\{\int d\nv\,v(\nv)\,{\bf f}^{j\dag}(\nv)\,{\rm SHT}^{-1}\left[\hat{\sf C}^{ab}_{\ell_1}{\rm SHT}\left(w\,{\bf g}^q\right)^{s_b}_{{\bf l}_1}\right]^{s_a}_{\nv}\right\}\tilde{\sf C}^{f^ig^p}_\ell
    \end{align}

    Once $\tilde{\sf C}^{ab}$ is calculated, it can be corrected for the effects of masking by inverting the linear transformation in Eq \ref{eq:master_transformation}. This transformation can be written explicitly by first transforming the power spectrum matrices into vectors $_v{\sf C}$. E.g. for $s_a=s_b=2$ we transform:
    \begin{equation}
      \hat{\sf C}^{ab}_\ell\equiv\left(
      \begin{array}{cc}
        C^{E_aE_b}_\ell & C^{E_aB_b}_\ell \\
        C^{B_aE_b}_\ell & C^{B_aB_b}_\ell
      \end{array}\right)
      \hspace{12pt}\text{into}\hspace{12pt}
      _v\hat{\sf C}^{ab}_\ell\equiv\left(
      \begin{array}{c}
        C^{E_aE_b}_\ell\\
        C^{E_aB_b}_\ell\\
        C^{B_aE_b}_\ell\\
        C^{B_aB_b}_\ell
      \end{array}\right).
    \end{equation}
    We can then write, in general:
    \begin{equation}\label{eq:cell_coupled}
      _v\tilde{\sf C}^{ab}_\ell=\sum_{\ell'}{\sf M}^{s_as_b}_{\ell\ell'}\cdot\,_v\hat{\sf C}^{ab}_{\ell'},
    \end{equation}
    where:
    \begin{align}
      &{\sf M}^{00}_{\ell\ell'}=\frac{2\ell'+1}{4\pi}\sum_{\ell''}(2\ell''+1)C^{vw}_{\ell''}\wtj{\ell}{\ell'}{\ell''}{0}{0}{0}^2\\
      &{\sf M}^{02}_{\ell\ell'}=M^{0+}_{\ell\ell'}\,\hat{\sf 1},
      \hspace{12pt}
      M^{0+}_{\ell\ell'}=\frac{2\ell'+1}{4\pi}\sum_{\ell''}(2\ell''+1)C^{vw}_{\ell''}\wtj{\ell}{\ell'}{\ell''}{0}{0}{0}\wtj{\ell}{\ell'}{\ell''}{2}{-2}{0}\\
      &{\sf M}^{22}_{\ell\ell'}=\left(
      \begin{array}{cccc}
        M^{++}_{\ell\ell'} &0 &0&M^{--}_{\ell\ell'}\\
        0&M^{++}_{\ell\ell'} &-M^{--}_{\ell\ell'}&0\\
        0&-M^{--}_{\ell\ell'} &M^{++}_{\ell\ell'}&0\\
        M^{--}_{\ell\ell'} &0 &0&M^{++}_{\ell\ell'}
      \end{array}\right)\\
      & \hspace{12pt}M^{\pm\pm}_{\ell\ell'}=\frac{2\ell'+1}{4\pi}\sum_{\ell''}(2\ell''+1)C^{vw}_{\ell''}\wtj{\ell}{\ell'}{\ell''}{2}{-2}{0}^2\frac{1\pm(-1)^{\ell+\ell'+\ell''}}{2}
    \end{align}
    
    If either the $B$ or $E$ modes of a spin-2 field has been purified, the equations above must be modified by carrying out the following modification in the equations above:
    \begin{align}
      &\wtj{\ell}{\ell'}{\ell''}{2}{-2}{0}\longrightarrow\\\nonumber
      &\wtj{\ell}{\ell'}{\ell''}{2}{-2}{0}+2\sqrt{\frac{(\ell+1)!(\ell-2)!(\ell''+1)!}{(\ell-1)!(\ell+2)!(\ell''-1)!}}\wtj{\ell}{\ell'}{\ell''}{1}{-2}{1}+\sqrt{\frac{(\ell-2)!(\ell''+2)!}{(\ell+2)!(\ell''-2)!}}\wtj{\ell}{\ell'}{\ell''}{0}{-2}{2}.
    \end{align}
    This change must be applied to the corresponding factors of $\wtj{\ell}{\ell'}{\ell''}{2}{-2}{0}$.


\subsection{Beam}
  Adding the effect of a beam amounts to redefining:
  \begin{equation}
    \mathsf{M}^{s_as_b}_{\ell_1\ell_2}\rightarrow\mathsf{M}^{s_as_b}_{\ell_1\ell_2}\,b^{ab}_{\ell_2},
  \end{equation}
  where $b^{ab}_\ell$ is the product of the harmonic transform of the beams for maps $a$ and $b$.
    
\subsection{Binning into bandpowers}
  Given the loss of information implicit in masking the originally full-sky field, it is in general not possible to invert Eq. \ref{eq:cell_coupled} directly. De usual approach to doing so is by binning the pseudo-$C_\ell$ into bandpowers. A bandpower $k$ is defined by a set of $N_k$ multipoles $\vec{\ell}_k\equiv(\ell_k^1,...,\ell_k^{N_k})$ and a set of weights $\vec{w}_k\equiv(w_k^1,...,w_k^{N_k})$ normalized such that $\sum_{i=1}^{N_k}w_k^i=1$. The $k-$th bandpower for the coupled pseudo-$C_\ell$ is then defined as:
  \begin{equation}
    _v\tilde{\sf B}^{ab}_k\equiv\sum_{i=1}^{N_k}w_k^i\,_v\tilde{\sf C}^{ab}_{\ell_k^i}
    =\sum_{i=1}^{N_k}w_k^i\sum_{\ell'}{\sf M}^{s_as_b}_{\ell_k^i\ell'}\,_v\hat{\sf C}^{ab}_\ell.
  \end{equation}
  One then proceeds by assuming that the true power spectrum is a step-wise function, taking constant values over the multipoles corresponding to each bandpower: $_v\hat{\sf C}^{ab}_\ell=\sum_k\,_v\hat{\sf B}^{ab}_k\Theta(\ell\in\vec{\ell}_k)$ (where $\Theta$ is a binary step function). The previous equation then reads:
  \begin{equation}
   _v\tilde{\sf B}^{ab}_k=\sum_{k'}\mathcal{M}^{s_as_b}_{kk'}\,_v\hat{\sf B}^{ab}_{kk'}\equiv\sum_{k'}\left(\sum_{\ell\in\vec{\ell}_k}\sum_{\ell'\in\vec{\ell}_{k'}}w_k^\ell{\sf M}^{s_as_b}_{\ell\ell'}\right)\,_v\hat{\sf B}^{ab}_{k'},
  \end{equation}
  which defines the binned coupling matrix $\mathcal{M}^{ab}_{kk'}$. The decoupled bandpowers are then estimated by inverting $\mathcal{M}^{ab}$:
  \begin{equation}
    _v\hat{\sf B}^{ab}_k=\sum_{k'}\left(\mathcal{M}^{ab}\right)^{-1}_{kk'}\,_v\tilde{\sf B}^{ab}_{k'}.
  \end{equation}

  Note that, even though this procedure is based on the assumption that the true power spectrum is step-wise constant, the bandpowers computed this way should be compared with the theoretical prediction subjected to the same type of transformation. I.e. the theoretical prediction for the bandpowers is:
  \begin{equation}
    _v\bar{\sf B}^{ab}_k=\sum_{k'}\left(\mathcal{M}^{ab}\right)^{-1}_{kk'}\sum_{\ell'\in\vec{\ell}_{k'}}w_{k'}^{\ell'}\sum_{\ell''}{\sf M}^{s_as_b}_{\ell'\ell''}\,_v\bar{\sf C}^{ab}_{\ell''},
  \end{equation}
  where we the overline $\bar{\,}$ denotes theoretical predictions.

\begin{thebibliography}{}
 \bibitem{2002ApJ...567....2H} Hivon, E., G{\'o}rski, K.~M., Netterfield, C.~B., et al.\ 2002, \apj, 567, 2 
 \bibitem{2017MNRAS.465.1847E} Elsner, F., Leistedt, B., \& Peiris, H.~V.\ 2017, \mnras, 465, 1847 
 \bibitem{2003ApJS..148..161K} Kogut, A., Spergel, D.~N., Barnes, C., et al.\ 2003, \apjs, 148, 161 
\end{thebibliography}

\end{document}


\subsection{Bandpowers}
Consider the case where you want to compute the power spectrum in band-powers given by
\begin{equation}
  \mathbf{B}^{ab}_k\equiv\frac{1}{N_k}\sum_{\ell=\ell_k}^{\ell_k+N_k-1} f(\ell)\,\mathbf{C}^{ab}_\ell,
\end{equation}
then Eq. \ref{eq:pcl_gen} above becomes
\begin{equation}\label{eq:pcl_bin}
  \langle\tilde{\mathbf{B}}^{ab}_k\rangle=\sum_{k'}\mathsf{M}^{B,s_as_b}_{k k'}\cdot\mathbf{B}^{ab}_{k'}+
  \langle\tilde{\mathsf{N}}^{B,ab}_k\rangle
\end{equation}
where the binned coupling matrix $\mathsf{M}^{B,s_as_b}$ is
\begin{equation}
  \mathsf{M}^{B,s_as_b}_{k_1,k_2}\equiv\frac{1}{N_{k_1}}\sum_{\ell_1=\ell_{k_1}}^{\ell_{k_1}+N_{k_1}-1}\sum_{\ell_2=\ell_{k_2}}^{\ell_{k_2}+N_{k_2}-1}\frac{f(\ell_1)}{f(\ell_2)}
  \mathsf{M}^{s_as_b}_{\ell_1\ell_2}
\end{equation}

\section{Mode deprojection}
Let ${\bf t}$ be a set of contaminant templates such that we can write:
\begin{equation}
  {\bf d}={\bf s}+{\bf n}+\hat{\bf t}\cdot{\bf a}.
\end{equation}
A maximum-likelihood estimate for ${\bf a}$ can be found as
\begin{equation}
 \tilde{\bf a}=(\hat{\bf t}^T\hat{\bf C}^{-1}\hat{\bf t})^{-1}\hat{\bf t}\hat{\bf C}^{-1}{\bf d}
\end{equation}


\section{Minimum variance quadratic estimator}

Let ${\bf d}_1$ and ${\bf d}_2$ be the data, given as a pixelized full-sky maps,
with covariance matrix
\begin{equation}
 \langle{\bf d}_1\,{\bf d}^T_1\rangle\equiv C^{12}(\hat{\mu})\equiv
 S^{12}(\hat{\mu})+N^{12}(\hat{\mu})=
 \sum_\ell C^{12}_\ell\,P_\ell(\hat{\mu})+N^{12}(\hat{\mu}),
\end{equation}
where $\hat{\mu}={\bf n}\,{\bf n}^T$, ${\bf n}$ is the vector of unit vectors
containing the angular coordinates to each pixel and $P_\ell(x)\equiv (2\ell+1)
L_\ell(x)/(4\pi)$, where $L_\ell$ are the Legendre polynomials. Note that we can
expand $P_\ell$ as
\begin{equation}
  P_\ell(\hat{\mu})\equiv\sum_{m=-\ell}^\ell
  Y_{\ell m}({\bf n})\cdot Y^\dag_{\ell m}({\bf n}).
\end{equation}

We will parametrize the power spectrum with a set of step functions, such that
\begin{equation}
 C^{12}_\ell=\sum_b c_b\,\Theta_b(\ell),\hspace{12pt}\leftarrow\hspace{12pt}
 \Theta_b(\ell)=1\,{\rm if}\,\ell\in[\ell^b_{\rm min},\ell^b_{\rm max}),
 \,\,0\,{\rm otherwise},
\end{equation}
thus we can write
\begin{equation}
  C^{12}(\hat{\mu})=\sum_b\,c_b\,Q_b(\hat{\mu})+N^{12}(\hat{\mu}),
  \hspace{6pt}{\rm with}\hspace{6pt}
  Q_b(\mu)=\sum_\ell\Theta_b(\ell)P_\ell(\mu).
\end{equation}

Let us write the most general quadratic estimator for the coefficients $c_b$:
\begin{equation}
 \tilde{c}_b\equiv {\bf d}^T_1\hat{E}_b{\bf d}_2-B_b.
\end{equation}
The mean value of $\tilde{c}_b$ is:
\begin{equation}
 \langle\tilde{c}_b\rangle=\sum_a c_a\Tr(\hat{E}_b\,\hat{Q}_a)+
 \Tr(\hat{E_b}\hat{N}^{12})-B_b,
\end{equation}
and therefore we can remove the noise bias by defining
\begin{equation}
  B_b\equiv\Tr(\hat{E_b}\hat{N}^{12}).
\end{equation}

The covariance matrix of this estimator is:
\begin{equation}
 \left\langle(\tilde{c}_a-\langle\tilde{c}_a\rangle)\,
 (\tilde{c}_b-\langle\tilde{c}_b\rangle)\right\rangle=
 \Tr(\hat{C}^{11}\hat{E}_a\hat{C}^{22}\hat{E}_b),
\end{equation}
where we have approximated $C_\ell^{12}\ll[C_\ell^{11},C_\ell^{22}]$. Minimizing
the variance of $\tilde{c}_a$ under the constrain that the coupling matrix
$W_{ab}\equiv\Tr(\hat{E}_b\,\hat{Q}_a)$ have unit diagonal, we find the following
minimization problem:
\begin{equation}
  L=\Tr\left[\hat{E}_b\hat{C}^{11}\hat{E}_b\hat{C}^{22}-
  2(\lambda\hat{E}_b\hat{Q}_b-1)\right],
\end{equation}
where $\lambda$ is the lagrange multiplier related to the constraint. The solution
to this problem is:
\begin{equation}
 \hat{E}_b=\frac{\hat{C}_{11}^{-1}\hat{Q}_b\hat{C}_{22}^{-1}}
 {\Tr(\hat{C}^{-1}_{11}\hat{Q}_b\hat{C}^{-1}_{22}\hat{Q}_b)}.
\end{equation}

Thus, a decoupled, unbiased and minimum-variance estimator, $\hat{c}_b$, for
$c_b$ can be found as:
\begin{equation}
 {\bf d}^T_1\hat{C}^{-1}_{11}\hat{Q}_b\hat{C}^{-1}_{22}{\bf d}_2-
 \Tr(\hat{C}^{-1}_{11}\hat{Q}_b\hat{C}^{-1}_{22}\hat{N})
 =\sum_a \hat{c}_a\Tr(\hat{C}^{-1}_{11}\hat{Q}_b\hat{C}^{-1}_{22}\hat{Q}_a).
\end{equation}
Note that this estimator requires a guess for the covariance matrix of the data
$C^{ij}(\hat{\mu})$, and exact minimum variance will only be attained for the true
covariance (which itself depends on the power spectrum coefficients $c_b$).
The choice of prior covariance will define the type of estimator. Note that this
formalism immediately encompasses cut-skies by setting to zero all elements of the
full covariance matrix involving unobserved pixels (this would correspond to the
limit of infinite uncorrelated noise for those pixels).
2002ApJ × .567 ×  × 2H}
\subsection{Strategies for the different terms}
\begin{itemize}
\item $\hat{C}^{-1}_{ii}\,{\bf d}_i$. This term can be computed by solving the
linear system:
\begin{equation}
  \hat{C}_{ii}{\bf z}_i={\bf d}_i
\end{equation}
via conjugate gradients. The action of $\hat{C}=\hat{S}+\hat{N}$ can be computed as
follows:
\begin{align}
 [\hat{S}{\bf z}](\nv_1)&\equiv\sum_{\nv_2}S(\nv_1\cdot\nv_2)z(\nv_2)\\
                        &=\frac{1}{\Delta\Omega}\int d\nv_2\sum_{\ell m}C_\ell
                          Y^*_{\ell m}(\nv_1)Y^*_{\ell m}(\nv_1)\,\tilde{z}(\nv_2)\\
                        &=\frac{1}{\Delta\Omega}\sum_{\ell m} Y^*_{\ell m}(\nv_1)
                        C_\ell \tilde{z}_{\ell m}\\
                        &=\frac{1}{\Delta\Omega}
{\rm SHT}^{-1}\left[C_\ell\,{\rm SHT}[\tilde{z}(\nv)]_{\ell m}\right],
\end{align}
where $\tilde{z}$ is the extension of $z$ to the whole sphere, with $\tilde{z}=0$ in
all unobserved pixels.

In most cases the noise power spectrum will be white, in which case $\hat{N}{\bf z}$
is just $\sum_{\nv}\sigma_N^2(\nv)z(\nv)$, with $\sigma^2(\nv)$ the per-pixel
variance.

Note that this procedure works for any $\hat{C}^{-1}{\bf v}$-type operation
(see below).

\item ${\bf d}^T_1\hat{C}^{-1}_{11}\hat{Q}_b\hat{C}^{-1}_{22}{\bf d}_2$. With
${\bf z}_i$ defined as above, this is just:
\begin{align}
 {\bf z}^T_i\hat{Q}_b{\bf z}_j=
 \frac{1}{(\Delta\Omega)^2}\sum_\ell\Theta_b(\ell)
 \sum_m(\tilde{z}^{i}_{\ell m})^*\tilde{z}^{i}_{\ell m}
\end{align}
\item $\Tr(\hat{C}^{-1}_{11}\hat{Q}_b\hat{C}^{-1}_{22}\hat{Q}_a)$. Let ${\bf v}$ be a random
vector with covariance $\langle {\bf v}{\bf v}^T\rangle=\hat{1}$, then one can
calculate traces by averaging over realizations of such
vectors:
 \begin{equation}
   \Tr{\hat{A}}=\left\langle {\bf v}^T\hat{A}{\bf v}\right\rangle.
 \end{equation}
 Then, the only thing to bear in mind in order to compute the trace above is that the action of the
 $\hat{Q}_a$ operator is:
 \begin{equation}
   \left(\hat{Q}_a{\bf v}\right)_{\nv}=\frac{1}{\Delta\Omega}{\rm SHT}^{-1}
   \left[\Theta_b(\ell){\rm SHT}\left[\tilde{v}\right]\right]_{\nv},
 \end{equation}
 where, as before $\tilde{v}$ is the extension of $v$ with zeros in all unobserved pixels.

 In order to minimize the number of operations needed to compute this trace we can compute
 it as:
 \begin{equation}
   \Tr(\hat{C}^{-1}_{11}\hat{Q}_b\hat{C}^{-1}_{22}\hat{Q}_a)=
   \left\langle(\hat{C}^{-1}_{11}\,{\bf v})^T\hat{Q}_b(\hat{C}^{-1}_{22}\hat{Q}_a\,{\bf v})\right\rangle
 \end{equation}

\end{itemize}

%\subsection{Computing $\langle\tilde{C}^N_l\rangle$}
%Consider the case where the window function is just a flat top-hat multiplied by the inverse variance of the noise:
%\begin{equation}
%  W(\nv)\equiv\Theta(\nv)\frac{\bar{\sigma}_N^2}{\sigma_N^2(\nv)},
%\end{equation}
%where $\sigma_N^2(\nv)$ is the variance per sterad at each point and $\bar{\sigma}^2_N$ is its sky-averaged value.
%
%For uncorrelated noise, we can write $N(\nv)$ as
%\begin{equation}
% N(\nv)=u(\nv)\,\sigma_N(\nv)
%\end{equation}
%where $u(\nv)$ is a white GRF with power spectrum $C^u_l=1$. In this case, the noise pseudo-$C_l$ can be estimated theoretically:
%\begin{equation}
% \langle\tilde{C}_l^N\rangle=\frac{1}{4\pi}\int d\Omega\,\sigma_N^2(\nv)W^2(\nv)=
% \frac{\bar{\sigma}_N^2}{4\pi}\int d\Omega\, \Theta(\nv)\frac{\bar{\sigma}_N^2}{\sigma_N^2(\nv)}=
% f_{\rm sky}\,\bar{\sigma}_N^2\overline{\left(\frac{\bar{\sigma}_N^2}{\sigma^2_N}\right)}
%\end{equation}
